\documentclass{article}

\usepackage{arxiv}

\usepackage[utf8]{inputenc} % allow utf-8 input
\usepackage[T1]{fontenc}    % use 8-bit T1 fonts
\usepackage{lmodern}        % https://github.com/rstudio/rticles/issues/343
\usepackage{hyperref}       % hyperlinks
\usepackage{url}            % simple URL typesetting
\usepackage{booktabs}       % professional-quality tables
\usepackage{amsfonts}       % blackboard math symbols
\usepackage{nicefrac}       % compact symbols for 1/2, etc.
\usepackage{microtype}      % microtypography
\usepackage{graphicx}

\title{Paper grupo Messi}

\author{
  }


% tightlist command for lists without linebreak
\providecommand{\tightlist}{%
  \setlength{\itemsep}{0pt}\setlength{\parskip}{0pt}}



\begin{document}
\maketitle


\begin{abstract}

\end{abstract}


\hypertarget{gemelos-digitales}{%
\subsection{Gemelos digitales}\label{gemelos-digitales}}

En esta seccion se estudia las aplicaciones de los gemelos digitales, en
nuestro caso la fuente de la cual sacamos informacion es la siguiente:

Bosque Peón, C. D. (2019). Los gemelos digitales en la industria 4.0.

El caso analizado es la utilizacion de los gemelos digitales en la
fuerza armada. Estos se aplicaron para hacer simulaciones de combate en
la fragata sin poner en riesgo los recursos. Ademas utilizaron la
impresion 3D para hacer las simulaciones, esto lo vincula con la
industria 4.0.

A continuacion se analizara la situacion del problema, luego se
planteara una hipotesis de la cual creemos que es la solucion y una
tesis de la solucion del problema con demostraciones al respecto

Situación del problema

La industria de Defensa se enfrenta al reto de desarrollar sistemas de
armas y vehículos complejos que sean eficaces, confiables y rentables.
Tradicionalmente, el diseño, la producción y el mantenimiento de estos
sistemas han sido procesos costosos y prolongados, que requieren de
pruebas físicas extensas y a menudo riesgosas.

Hipótesis

La tecnología de gemelos digitales se presenta como una solución
potencial a estos desafíos. Los gemelos digitales son representaciones
virtuales precisas de sistemas físicos que pueden usarse para simular y
analizar su comportamiento en diferentes escenarios.

Tesis

Los gemelos digitales pueden revolucionar la industria de Defensa al
permitir:

Diseño y producción más eficientes: Los gemelos digitales pueden usarse
para probar y optimizar diseños virtualmente, antes de construir
prototipos físicos. Esto puede reducir significativamente el tiempo y el
costo de desarrollo. Mayor confiabilidad: Los gemelos digitales pueden
usarse para simular condiciones extremas y fallas, lo que permite
identificar y solucionar problemas de diseño antes de que afecten al
sistema real. Mantenimiento predictivo: Los gemelos digitales pueden
usarse para monitorear el estado de los sistemas en tiempo real y
predecir fallas potenciales, lo que permite un mantenimiento preventivo
más efectivo y reduce el tiempo de inactividad. Mejor toma de
decisiones: Los gemelos digitales pueden usarse para generar datos y
análisis que pueden ayudar a los operadores a tomar mejores decisiones
sobre el uso y mantenimiento de los sistemas. Demostración

El texto proporcionado describe dos ejemplos concretos de cómo se están
utilizando los gemelos digitales en la industria de Defensa:

Fragata F-110: Navantia está utilizando gemelos digitales para diseñar y
desarrollar la fragata F-110 para la Armada española. Los gemelos
digitales están permitiendo a la empresa simular diferentes escenarios
operativos y evaluar el rendimiento del buque en condiciones extremas.
Esto está ayudando a optimizar el diseño del buque y a identificar y
solucionar problemas potenciales antes de que afecten al buque real.
Vehículo de combate sobre ruedas 8x8 Dragón: Indra está utilizando
gemelos digitales para desarrollar el vehículo de combate sobre ruedas
8x8 Dragón para el Ejército de Tierra español. Los gemelos digitales
están permitiendo a la empresa simular diferentes escenarios de combate
y evaluar el rendimiento del vehículo en diferentes condiciones. Esto
está ayudando a optimizar el diseño del vehículo y a garantizar que
cumpla con los requisitos del Ejército de Tierra. Estos ejemplos
demuestran el potencial de los gemelos digitales para transformar la
industria de Defensa. A medida que la tecnología continúa
desarrollándose, es probable que los gemelos digitales se vuelvan aún
más importantes para el diseño, la producción y el mantenimiento de
sistemas de armas y vehículos complejos.

Conclusión

Los gemelos digitales tienen el potencial de revolucionar la industria
de Defensa al permitir el diseño, la producción y el mantenimiento más
eficientes, confiables y rentables de sistemas de armas y vehículos
complejos. A medida que la tecnología continúa desarrollándose, es
probable que los gemelos digitales desempeñen un papel cada vez más
importante en el futuro de la Defensa.

\bibliographystyle{unsrt}
\bibliography{}


\end{document}
